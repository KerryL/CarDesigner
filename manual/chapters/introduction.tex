\chapter{Introduction} \label{ch:introduction}

%\vvase{} was designed as a tool that could be useful throughout the process of designing and testing a vehicle.  The vision was a single product that could perform kinematic analysis, dynamic simulation, process tire data and replay recorded data.

\vvase{} is a tool for analyzing suspension kinematics.  Although there are other tools available that can also help analyze suspension kinematics, \vvase{} provides a unique facility for performing complex optimizations.

What does it do (outputs vs inputs).  Briefly describe output pane, iterations and GAs.

Currently, \vvase{} is limited to cars employing a double A-arm suspension at both ends.  It supports several spring/damper attachment methods, U-bar and T-bar anti-roll bars, asymmetric suspensions and several other configuration options that allow virtually any double A-arm suspension to be modeled.

Often times, engineering problems 

software is very powerful, can solve any problem, but only for expert users.  Hard to be confident that input values are correct; hard to be confident outputs are interpreted correctly.

The types of problems facing engineers are very broad.

making better/faster decisions


% Purpose of software
% Design goals
% Types of problems it can help with
% How is it different than other products
% Limitations
% License?